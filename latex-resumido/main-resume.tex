\documentclass[a4paper,11pt]{article}

\usepackage[brazil]{babel}
\usepackage[utf8]{inputenc}
\usepackage{times}
\usepackage[T1]{fontenc}
\usepackage{amsmath,amssymb}
\usepackage{url}
\usepackage[square]{natbib}
\usepackage{indentfirst}
\usepackage{fancyhdr}
\usepackage{graphicx}
\usepackage{float}
\usepackage{booktabs,array}
\usepackage{algorithm}
\usepackage{algpseudocode}
\usepackage{multicol}
\usepackage{enumitem}
\usepackage{geometry}
\geometry{
  a4paper,
  left=2.9cm,
  right=2.9cm,
  top=3.3cm,
  bottom=2.5cm
}

\pagestyle{fancy}
\fancyhf{}
% Se desejar, adicione o cabeçalho do evento:
\fancyhead[C]{\includegraphics[scale=0.32]{sbpo2025-header-logo.png}}
\renewcommand{\headruleskip}{-1mm}
\setlength\headheight{86pt}
\addtolength{\textheight}{-86pt}
\setlength{\headsep}{5mm}
\setlength{\footskip}{4.08003pt}

\setlength{\parindent}{0pt}
\setlength{\parskip}{5pt}
\textwidth 13.5cm
\textheight 19.5cm
\columnsep .5cm

\title{Otimização da Observação Ambiental em APAs com Sentinel-2: Integração entre Heurística e Programação Inteira Mista}

\author{
Luryan Delevati Dorneles$^{1*}$ \\
Prof. Dr. Rian Pinheiro$^{2}$ \\
Prof. Dr. Bruno Nogueira$^{3}$
}

\date{}
\begin{document}

\maketitle

\vspace{-0.5cm}

\begin{center}
{\footnotesize
*Autor correspondente\\
$^1$ Instituto de Computação, Universidade Federal de Alagoas (UFAL), Maceió, Brasil. Email: ldd@ic.ufal.br / ORCID: 0000-0000-0000\\
$^2$ Instituto de Computação, Universidade Federal de Alagoas (UFAL), Maceió, Brasil. Email: rian@ic.ufal.br\\
$^3$ Instituto de Computação, Universidade Federal de Alagoas (UFAL), Maceió, Brasil. Email: bruno@ic.ufal.br
}
\end{center}

\bigskip
\noindent
{\small{\bf Palavras-chave:} Heurística Construtiva Gulosa. Otimização Híbrida. Sensoriamento Remoto. Programação Linear Inteira Mista. Sentinel-2.}

\vspace{8mm}

\begin{abstract}
This paper presents a hybrid approach to optimize Sentinel-2 image coverage over Environmental Protection Areas (APAs) in Alagoas, Brazil. The methodology combines a greedy constructive heuristic with a multi-criteria evaluation function to identify potential Sentinel-2 mosaic groups, followed by an exact optimization using Mixed-Integer Linear Programming (MILP). The goal is to maximize the qualified useful coverage of monitored areas, considering cloud cover, valid pixels, and orbit compatibility. The MILP model incorporates a proxy cost to penalize the use of Sentinel-1 (SAR) fallback. Results show that the hybrid approach yields high-quality solutions, balancing computational efficiency and optimality in selecting Sentinel-2 imagery.
\end{abstract}

\bigskip
\noindent
{\small{\bf Keywords:} Greedy Constructive Heuristic. Hybrid Optimization. Remote Sensing. Mixed-Integer Linear Programming. Sentinel-2.}

\newpage

\section{Introdução}
O monitoramento eficiente de Áreas de Proteção Ambiental (APAs) é fundamental para a conservação dos recursos naturais. Este estudo foca nas APAs de Murici, Santa Rita, Catolé e Fernão Velho, e Pratagy, em Alagoas, que somam cerca de 163.956 hectares. O objetivo é otimizar a seleção de imagens Sentinel-2 (ópticas) para maximizar a cobertura útil dessas áreas entre 13 de abril de 2024 e 13 de abril de 2025, considerando limitações climáticas, principalmente a alta cobertura de nuvens, comum na região costeira do estado.

Além da cobertura de nuvens, fatores como qualidade intrínseca dos pixels e sobreposição entre cenas adjacentes são críticos para a composição de mosaicos úteis. Para lidar com isso, o problema foi formalizado como um modelo de Programação Linear Inteira Mista (MILP) multiobjetivo, visando simultaneamente maximizar a cobertura espacial e minimizar nuvens e custo de aquisição, considerando o uso complementar de imagens Sentinel-1 (SAR) como penalidade adicional.

\section{Objetivos}
Este trabalho tem como objetivo otimizar a seleção de imagens Sentinel-2 para as APAs de Alagoas, por meio de uma abordagem híbrida que:

\begin{itemize}
    \item Gera combinações iniciais de imagens utilizando heurística construtiva gulosa com avaliação multicritério (cobertura geográfica, qualidade dos pixels, nuvens e compatibilidade orbital);
    \item Aplica um modelo MILP para selecionar a combinação final de mosaicos, incorporando o uso complementar de Sentinel-1 (SAR) como custo proxy para compensar limitações ópticas.
\end{itemize}

\section{Caracterização das Imagens}
Os satélites Sentinel-1 (Radar de Abertura Sintética - SAR) e Sentinel-2 (multiespectral/óptico), componentes do programa Copernicus da Agência Espacial Europeia (ESA), oferecem capacidades complementares para o monitoramento ambiental. Sentinel-2 fornece imagens ópticas de alta resolução espacial (até 10 metros) em 13 bandas espectrais, ideais para análise de vegetação e uso do solo, mas sua aquisição é impedida pela cobertura de nuvens. Sentinel-1, com seu sensor SAR, opera independentemente das condições atmosféricas e de iluminação, fornecendo dados valiosos em regiões com nebulosidade frequente, embora com características de resolução e interpretação distintas das imagens ópticas.

\section{Metodologia}

\subsection{Abordagem Híbrida de Otimização}
Nossa metodologia consiste em duas fases para otimizar a seleção de imagens Sentinel-2:

\begin{itemize}
    \item \textbf{Fase 1: Heurística Construtiva Gulosa} - Para reduzir a complexidade combinatória, usamos uma heurística construtiva gulosa baseada em fatores de qualidade extraídos dos metadados (data, nuvens e cobertura geográfica). Li, Liu \& Liu propuseram o DD‑RSIRA, que gera uma solução inicial ponderando data de aquisição, \texttt{CLOUDY\_PIXEL\_PERCENTAGE} e extensão territorial, seguida de refinamento por busca local com esquema ganho‑perda. Zhang et al. apresentaram um método exato para extrair subconjuntos mínimos de um acervo altamente redundante, formulando o problema como cobertura de conjuntos e resolvendo-o de forma ótima.
    \item \textbf{Fase 2: Otimização Exata via MILP} - Com a lista de candidatos reduzida pela heurística, adotamos um modelo MILP com variáveis binárias para cada grupo de cenas, impondo restrições de cobertura geográfica, limite máximo de nuvens e orçamento. Essa formulação segue diretamente Combarro Simón et al. e incorpora conceitos de agendamento de janelas temporais e transição de atitude de Kim et al.
\end{itemize}

\subsection{Notação Matemática Relevante}
Os principais parâmetros e variáveis do modelo incluem:
\begin{multicols}{2}
    \begin{itemize}[leftmargin=*,noitemsep,topsep=0pt]
        \item $C_i$: Cobertura efetiva de cada imagem 
        \item $C_g$: Cobertura geográfica de uma imagem
        \item $P_x$: \% Pixels válidos de cada imagem 
        \item $C_n$: Cobertura geográfica normalizada
        \item $Q$: Fator de qualidade da imagem
        \item $C_c$: \% Cobertura de nuvens
        \item $G$: Conjunto de grupos de mosaico candidatos identificados pela heurística
        \item $y_g$: Variável binária indicando se o grupo $g$ é selecionado (1) ou não (0)
        \item $E_g$: Cobertura geográfica estimada do grupo $g$
        \item $Q_g$: Fator de qualidade médio do grupo $g$
        \item $\beta$: Peso para penalizar a seleção de qualquer grupo (proxy para custo SAR)
    \end{itemize}
\end{multicols}

\subsection{Fase 1: Heurística Construtiva Gulosa}

\subsubsection{Classificação de Imagens}
As imagens Sentinel-2 são classificadas como "centrais" ou "complementos" com base na cobertura efetiva:
\begin{equation}
C_i = C_g \times P_x
\end{equation}

\subsubsection{Função de Avaliação Multi-Critério}
A qualidade de uma imagem candidata é avaliada pela função de efetividade $E$:
\begin{equation}
E = C_n \times Q
\end{equation}
Onde $C_n$ é a cobertura geográfica normalizada (fração da APA coberta pela imagem) e $Q$ é o fator de qualidade:
\begin{align}
Q = (1 - C_c) \times P_x
\end{align}

\subsubsection{Identificação de Grupos de Mosaico}
A heurística busca maximizar a cobertura útil e a qualidade dos mosaicos, formando grupos de imagens candidatos a serem selecionados na fase MILP. Cada grupo $g$ recebe métricas $E_g$ (cobertura estimada) e $Q_g$ (qualidade média).

\begin{algorithm}[H]
\caption{Heurística Construtiva Gulosa para Seleção de Mosaicos Sentinel-2}
\begin{algorithmic}[1]
\Require Conjunto de imagens candidatas $I$, área de interesse $A$
\Ensure Grupos de mosaico candidatos $G$
\State $G \gets \emptyset$
\While{há regiões de $A$ não cobertas}
    \ForAll{imagem $i \in I$}
        \State Calcule $C_i = C_g \times P_x$
        \State Calcule $Q_i = (1 - C_c) \times P_x$
        \State Calcule $E_i = C_n \times Q_i$
    \EndFor
    \State Selecione $i^* = \arg\max_{i} E_i$ entre as imagens não selecionadas
    \State Adicione $i^*$ ao grupo corrente $g$
    \State Atualize a região coberta de $A$ e remova $i^*$ de $I$
    \If{grupo $g$ cobre suficientemente $A$}
        \State Adicione $g$ a $G$ e inicie novo grupo
    \EndIf
\EndWhile
\State \Return $G$
\end{algorithmic}
\end{algorithm}

\subsection{Fase 2: Modelo de Programação Linear Inteira Mista}
\subsubsection{Função Objetivo}
\begin{equation}
\max \left( \sum_{g \in G} (E_g \times Q_g \times y_g) - \alpha \sum_{g \in G} y_g - \beta \sum_{g \in G} y_g \right)
\end{equation}

\subsubsection{Restrições Principais}
\begin{align}
    \sum_{g \in G} y_g &\leq N_{\max} \\
    \sum_{g \in G : i \in img(g)} y_g &\leq 1, \quad \forall i \in I'
\end{align}

\section{Resultados}
Observou-se que, com os parâmetros estimados diretamente, o espaço viável definido pelas restrições é bastante restrito, dificultando a obtenção de soluções ótimas viáveis. Para fins de análise geométrica e visualização da estrutura do modelo, aplicou-se um relaxamento controlado das restrições, permitindo identificar o poliedro viável ampliado. Essa abordagem não interfere nos resultados obtidos via programação inteira mista, sendo utilizada apenas para fins exploratórios e de análise da sensibilidade do modelo.

\begin{figure}[H]
    \centering
    \includegraphics[width=0.5\textwidth]{optimization_visualization.png}
    \caption{
        Região viável considerando relaxamento de 0{,}33 nas restrições:
        $y_1 + y_2 \leq E_1 + 0{,}33$, 
        $y_1 \geq Q_1 - 0{,}33$, 
        $y_2 \leq E_2 + 0{,}33$, 
        $y_2 \geq Q_2 - 0{,}33$.
    }
    \label{fig:areas}
\end{figure}

A aplicação da metodologia híbrida resultou na seleção otimizada de grupos de mosaicos Sentinel-2 para a cobertura das APAs estudadas. A fase heurística identificou potenciais combinações de imagens, e a subsequente otimização via MILP selecionou o conjunto final.

\subsection{Estatísticas do Processamento e Otimização}
\begin{itemize}
    \item \textbf{Total de imagens Sentinel-2 consideradas no período:} 310
    \item \textbf{Potenciais combinações de mosaicos identificadas pela heurística:} 40
    \item \textbf{Total de grupos de mosaico selecionados pela otimização MILP:} 24
\end{itemize}

A otimização MILP, implementada com o solver CPLEX e utilizando os parâmetros $\alpha=0.05$ e $\beta=0.1$, selecionou 24 grupos de mosaicos. Este resultado representa a solução que maximiza a função objetivo, equilibrando a cobertura geográfica estimada ($E_g$) e o fator de qualidade ($Q_g$) dos grupos selecionados, ao mesmo tempo que aplica penalidades pelo número de grupos e pelo custo implícito associado ao fallback para SAR (modelado por $\beta$).

\section{Conclusão}
Este artigo apresentou uma abordagem híbrida para otimização da cobertura de imagens Sentinel-2 em APAs de Alagoas, combinando uma heurística construtiva gulosa com função de avaliação multicritério e um modelo MILP. Esta metodologia em duas fases permite lidar com a complexidade combinatória do problema: a heurística identifica eficientemente grupos potenciais de imagens para mosaicos, e o MILP otimiza a seleção final desses grupos pré-definidos.

Os resultados preliminares indicam que a abordagem híbrida proporciona soluções de alta qualidade, equilibrando eficiência computacional e otimalidade. A heurística consegue identificar eficientemente grupos de imagens compatíveis, e o modelo MILP complementa esta análise com uma otimização global, considerando as restrições e guiado por penalidades ajustáveis que permitem modelar trade-offs como o desincentivo ao uso implícito de dados SAR (representado pelo parâmetro $\beta$).

\end{document}